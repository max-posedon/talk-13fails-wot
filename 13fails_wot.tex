\documentclass{beamer}

\mode<presentation>
{
  \usetheme{Warsaw}
  \useoutertheme{miniframes}
  \setbeamercovered{transparent}
}

\usepackage[english,russian]{babel}
\usepackage[utf8]{inputenc}
\usepackage[T1]{fontenc}

\title[World of Tanks: Мировая Война]{World of Tanks\\13 "фейлов" при разработке Мировой Войны}
\author{Максим 'max\_posedon' Мельников}
\institute{Wargaming.net}
\date[КРИ 2011]{Конференция Разработчиков Игр 2011}

\begin{document}

\begin{frame}
  \titlepage
\end{frame}

\begin{frame}{Содержание}
  \tableofcontents
  % You might wish to add the option [pausesections]
\end{frame}

\section{Архитектура}
\begin{frame}{1. Мировая Война в World of Tanks}
    \begin{block}{Некоторые факты о Мировой Войне}
        \begin{itemize}
            \item{пошаговая стратегия, про передвижение фишек, сражения с противником и захватом территорий}
            \item{МВ работает в браузере, а бои в клиенте}
            \item{новая версия: расширение в 4 раза}
        \end{itemize}
    \end{block}

    \pause
    \begin{block}{Глоссарий}
        \begin{itemize}
            \item{движок - сервер+клиент, основанные на BigWorld}
            \item{клиент - клиентская часть WOT}
            \item{сервер - серверная часть WOT}
            \item{web - web составляющая игры: Мировая Война, турниры}
        \end{itemize}
    \end{block}
\end{frame}

\begin{frame}{2. Движок игры как чёрный ящик}
    \begin{block}{World of Tanks и Мировая Война}
        \begin{itemize}
            \item{независимость}
            \item{доступность}
            \item{минимизация рисков}
        \end{itemize}
    \end{block}
     
    \pause
    \begin{block}{LAMP}
        \begin{itemize}
            \item просто
            \item стабильно
            \item огромный опыт
        \end{itemize}
    \end{block}
\end{frame}

\begin{frame}{3. Экспорт данных}
    \begin{columns}
        \column{.45\textwidth}
            \begin{block}{BigWorld}
                \begin{itemize}
                    \item{аккаунты}
                    \item{кланы}
                    \item{результаты боёв}
                \end{itemize}
            \end{block}

            \pause
            \begin{block}{AMQP}
                \begin{itemize}
                    \item{RabbitMQ}
                    \item{доработка движка}
                    \item{осенью отказались}
                    \item{весной реализовали}
                \end{itemize}
            \end{block}
 
        \column{.45\textwidth}
            \pause
            \begin{block}{MySQL}
                \begin{enumerate}
                    \item{BigWorld}
                    \item{MySQL(export\_queue)}
                    \item{RabbitMQ}
                    \item{обработка}
                    \item{MySQL(highlevel\_*)}
                    \item{расчёт рейтингов}
                \end{enumerate}
            \end{block}
    \end{columns}
\end{frame}

\begin{frame}{4. Управление движком}
    \begin{block}{Сервер}
        \begin{itemize}
            \item управление аккаунтом
            \item управление кланом
            \item создание боёв
        \end{itemize}
    \end{block}

    \pause
    \begin{block}{BigWorld WebIntegration}
        \begin{itemize}
            \item синхронный подход
        \end{itemize}
    \end{block}

    \pause
    \begin{block}{AMQP}
        \begin{itemize}
            \item асинхронный подход
        \end{itemize}
    \end{block}
\end{frame}

\section{Проблемы и их решения}
\begin{frame}{5. FAIL: синхронизация данных в движке и в web-е}
    \begin{block}{Информация о пользователе}
        \begin{itemize}
            \item{обновляется при выходе из клиента}
            \item{не обновлялась при платеже}
        \end{itemize}
    \end{block}
    
    \begin{columns}
        \pause
        \column{.45\textwidth}
            \begin{block}{Старый подход}
                \begin{enumerate}
                    \item{приходит платёж}
                    \item{сохраняется в таблице}
                    \item{пользователь заходит}
                    \item{начисляется золото}
                    \item{пользователь выходит}
                    \item{экспорт данных}
                \end{enumerate}
            \end{block}

        \column{.45\textwidth}
            \pause
            \begin{block}{Новый подход}
                 \begin{enumerate}
                    \item{приходит платёж}
                    \item{сохраняется в таблице}
                    \item{начисляется золото}
                    \item{экспорт данных}
                \end{enumerate}
             \end{block}

            \pause
            \begin{block}{Опасность}
                \begin{itemize}
                    \item{DDOS}
                \end{itemize}
            \end{block}
    \end{columns}
\end{frame}

\begin{frame}{6. FAIL: Боевое API не готово}
    \begin{block}{Особенности разработки}
        \begin{itemize}
            \item{Мировая Война разрабатывается независимо}
            \item{функционал МВ не зависит от сервера}
        \end{itemize}
    \end{block}
    
    \begin{columns}
        \column{.45\textwidth}
            \pause
            \begin{block}{Обходной путь}
                \begin{itemize}
                    \item{турниры в ручном режиме}
                    \item{альфа тест Мировой Войны без BigWorld}
                \end{itemize}
            \end{block}

        \column{.45\textwidth}
            \pause
            \begin{block}{Решение}
                \begin{enumerate}
                    \item{API готов в декабре}
                    \item{интеграция Мировой Войны с BigWorld}
                \end{enumerate}
             \end{block}
    \end{columns}
\end{frame}

\begin{frame}{7. Мировая Война: используемые технологии}
    \begin{block}{Languages \& Tools}
        \begin{itemize}
            \item{OS: Linux (Centos)}
            \item{Web Server: nginx, mod\_wsgi(apache)}
            \item{Language: Python, Django, amqplib, jinja2, PIL, ...}
            \item{Storage: MySQL, RabbitMQ, memcached}
            \item{Web: HTML, CSS, jQuery, SVG}
            \item{BigWorld: Web Integration (python)}
            \item{...}
        \end{itemize}
    \end{block}
\end{frame}

\begin{frame}{8. FAIL: 504 Gateway Timeout}
    \begin{columns}
        \column{.475\textwidth}
            \begin{block}{Нагрузка на сервер}
                \begin{itemize}
                    \item{бои: 700/день}
                    \item{участники: 20'000/день}
                    \item{посетители: 50'000/день}
                    \item{посещения: 100'000/день}
                    \item{страницы: 465'000/день}
                    \item{динамика: 1'500'000/день}
                    \item{статика: 15'000'000/день}
                \end{itemize}
            \end{block}

        \pause
        \column{.575\textwidth}
            \begin{block}{Архитектура}
                \begin{itemize}
                    \item{frontend: nginx}
                    \item{nginx: workers=24}
                    \item{nginx: epoll, sendfile}
                    \item{nginx: gzip backend}
                    \item{backend: Apache(mod\_wsgi)}
                    \item{mod\_wsgi: Django}
                    \item{mod\_wsgi: workers=24}
                    \item{Django -> MySQL, RabbitMQ}
                    \item{RabbitMQ, MySQL -> BigWorld}
                \end{itemize}
            \end{block}
    \end{columns}
\end{frame}

\begin{frame}{9. FAIL: надежда на стабильность движка}
    \begin{block}{Старый подход}
        \begin{itemize}
            \item{отсутствие корректной обработки исключительных ситуаций в движке}
        \end{itemize}
    \end{block}

    \pause
    \begin{block}{Новый подход}
        \begin{itemize}
            \item{полная изоляция всех запросов в движок}
            \item{retry, везде где это возможно и имеет смысл}
            \item{самоконтроль и самовосстановление системы}
        \end{itemize}
    \end{block}
\end{frame}

\begin{frame}{10. FAIL: не уследили за ростом}
    \begin{block}{Ошибки}
        \begin{itemize}
            \item{не угадали скорость роста нагрузки}
        \end{itemize}
    \end{block}

    \pause
    \begin{block}{Выводы}
        \begin{itemize}
            \item{при реализации нужно накапливать данные о том, какие компоненты выдерживают какие нагрузки}
            \item{планировать 10 кратный рост}
        \end{itemize}
    \end{block}
\end{frame}


\begin{frame}{11. FAIL: multi-язычность}
    \begin{block}{Проблемы локализации Мировой Войны}
        \begin{itemize}
            \item{один сервер - много языков}
            \item{описания боёв на родном языке для клиенте}
            \item{движок игры не должен знать про web}
        \end{itemize}
    \end{block}

    \pause
    \begin{block}{Используемый подход}
        \begin{itemize}
            \item{gettext в клиенте и в web-е}
            \item{веб содержит все тексты и локализационные файлы}
            \item{на сервер передаются тексты на всех языках}
            \item{клиент получает данные тока на одном языке}
        \end{itemize}
    \end{block}
\end{frame}


\begin{frame}{12. FAIL: Интернациолизация}
    \begin{block}{Проблемы интернациолизации Мировой Войны}
        \begin{itemize}
            \item{несколько серверов - различные пользователи}
            \item{несколько часовых поясов на сервер}
        \end{itemize}
    \end{block}

    \pause
    \begin{block}{Решение проблемы}
        \begin{itemize}
            \item{добавить поддержку часовых поясов}
            \item{разделить мир на регионы}
            \item{общие данные на всех серверах}
            \item{опциональное отключение регионов}
        \end{itemize}
    \end{block}
\end{frame}

\section{Дальнейшее развитие}
\begin{frame}{13. Будущее Мировой Войны}
    \begin{block}{Пользователи хотят}
        \begin{itemize}
            \item{дипломатию}
            \item{экономику}
            \item{наёмников}
        \end{itemize}
    \end{block}

    \pause
    \begin{block}{Пользователи получат}
        \begin{itemize}
            \item{fun, какой именно - будет ясно после расширения}
        \end{itemize}
    \end{block}

    \pause
    \begin{block}{Программисты получат}
        \begin{itemize}
            \item{интересный высоконагруженный проект}
        \end{itemize}
    \end{block}
\end{frame}

\begin{frame}{}
    \begin{block}{Вопросы?}
    Максим Мельников <m\_melnikau@wargaming.net>
    \end{block}
\end{frame}

\end{document}
